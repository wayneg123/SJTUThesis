%# -*- coding: utf-8-unix -*-
%%==================================================
%% abstract.tex for UJS Master Thesis
%%==================================================

\begin{abstract}

江苏大学是2001年8月经教育部批准,由原江苏理工大学、镇江医学院、镇江师范专科学校合并组建。原江苏理工大学的前身镇江农业机械学院,1960年由南京工学院(现东南大学)分设独立建校,1978年被国务院确定为全国88所重点大学之一,1981年成为全国首批具有博士、硕士、学士学位授予权的高校,1982年及1994年先后更名为江苏工学院和江苏理工大学,综合实力一直位居全国高校百强之列。学校坐落在风景秀丽的国家历史文化名城——江苏省镇江市,办学历史可追溯到1902年刘坤一、张之洞等在南京创办的三江师范学堂。 

 学校学科门类齐全,涵盖工学、理学、医学、管理学、经济学、法学、文学、历史学、教育学、艺术学等10大学科门类。设有24个学院,88个本科专业;学校博士学位授权点涵盖13个一级学科,设有13个博士后科研流动站;硕士学位授权点涵盖41个一级学科,有11个硕士专业学位类别,26个工程硕士授权领域。现有2个国家重点学科,1个国家重点(培育)学科,6个江苏高校优势学科。学校的工程学、材料科学、临床医学、化学和农业科学5个学科进入ESI排名全球前1\%(并列全国高校第34位)。2015英国《泰晤士高等教育》第三届金砖国家和新兴经济体大学排名,学校列第181位。中国管理科学研究院《2016中国大学评价》,学校综合排名列全国高校第48位。
 
校园占地面积2822亩,各类建筑面积128余万平方米。教学科研仪器设备总值7.6亿元。图书馆建筑面积5.1万平方米,藏书305余万册,电子图书173余万册,建有教育部科技查新工作站。拥有一所集医疗、教育、科研、预防为一体的三级甲等附属医院。设有江苏大学出版社。

学校现有教职工5763人(含直属附属医院),其中专任教师2475人(教授450余人,具有博士学位的比例达54\%,具有海外经历的比例达24\%),集聚了一批以国家“千人计划”、“长江学者”、国家杰青、国家万人计划领军人才、国家“青年千人计划”为代表的高层次人才群体。在校生33000余人,其中全日制本科生22000余人,研究生10000余人,学历留学生1000余人。江苏大学京江学院全日制在校生近10000人。  
学校是“中央与地方共建高校”、“全国本科教学工作水平优秀高校”、全国“卓越工程师培养计划”首批试点高校、“国家教育体制改革试点高校”、“中国政府奖学金来华留学生接受高校”。近年来,学校获国家级教学成果奖5项,形成了一批以国家特色专业、国家级精品课程、国家精品视频公开课、国家精品资源共享课、国家实验教学示范中心、国家优秀教学团队为代表的优质教学资源;获评“全国百篇优秀博士学位论文”3篇,提名奖5篇;“挑战杯”全国大学生课外学术科技作品竞赛连续5届喜捧“优胜杯”,校大学生男子排球队屡获全国冠军,女子沙滩排球队获世界大学生运动会第7名,女子足球队获世界大学生“五人制”足球锦标赛季军;毕业生就业率一直保持在95\%以上,学校是“全国高校毕业生就业工作先进集体”、“全国50所毕业生就业典型经验高校”和首批“全国高校实践育人创新创业基地”高校和“江苏省大学生创业教育示范学校”。

“十二五”以来,学校科研经费总量达27.89亿元,其中纵向经费8.56亿元;获批国家自然基金项目709项(2014、2015年均居全国高校前50位),其中国家973、863计划等重大重点科技项目52项;获批国家社科基金项目39项;SCI检索收录论文5231篇;截至目前,学校共获得国家级科技成果奖8项、何梁何利基金科学与技术创新奖2项、国家杰出青年基金项目2项;拥有国家水泵及系统工程技术研究中心、混合动力车辆国家地方联合工程中心、国家级新农村发展研究院等3个国家级科技创新平台;建有国家知识产权培训(江苏)基地。学校牵头成立的现代农业装备与技术协同创新中心被认定为江苏省首批高校协同创新中心。2015年度,学校发明专利授权量居全国高校第6位。
学校先后与美国、英国、德国、澳大利亚、奥地利、日本等国家的117所高校及科研机构建立了长期合作关系,与奥地利格拉茨大学共建了孔子学院和汉德语言文化中心。与德国马格德堡大学合作举办了工程热物理专业硕士研究生教育项目,与美国阿卡迪亚大学合作开办了数学与应用数学专业本科教育项目。
学校坚持以人为本,大力推进和谐校园、民主法治和校园文化建设,党建创新不断加强。学校党委被中央组织部表彰为全国创先争优先进基层党组织,连续两次被评为江苏省“高校先进基层党组织”。学校多次获江苏省文明单位、和谐校园、平安校园等荣誉称号。

目前,江苏大学正高举中国特色社会主义伟大旗帜,深入贯彻党的十八大和习近平总书记系列重要讲话精神,紧紧围绕学校第三次党代会确定的奋斗目标,汇聚全校师生的智慧和力量,坚定不移地走以提升质量、强化特色为核心的内涵式发展道路,为把学校早日建成“高水平、有特色、国际化研究型大学”而努力奋斗!

\keywords{\large 江苏大学 \quad 博学 \quad 求是 \quad 明德}
\end{abstract}

\begin{englishabstract}

Jiangsu University (JSU) was founded in 1902 as a part of Sanjiang Normal University. It was retitled as Jiangsu University by integrating Jiangsu University of Science and Technology, Zhenjiang Medical College and Zhenjiang Teachers’ College with the approval of the Ministry of Education of China in August, 2001. The university’s undergraduate teaching was graded excellent by the Ministry of Education in 2004. It has developed to be a national comprehensive key university. According to the Evaluations of China’s Universities in 2017 by China Academy of Management Science, JSU is ranked 41. It is committed to cultivating talents with 4C (Confidence, Communication, Cooperation and Creation). Now the university is launching the new orientation of schooling for high-level, research-oriented university with strength of engineering and strategy of internationalization.
 
JSU offers 88 undergraduate programs, 170 master programs, and 42 PhD programs in 10 academic fields: Engineering, Science, Management, Economics, Medicine, Law, Education, Literature, Art and History. The university has 13 post-doctoral research stations. Distinguished among its peers for its academic rigor, the 24 schools are competing and collaborating with each other for a high-level, research-oriented university with distinctive features and internationalization strategy.

JSU has 5,763 staff members (including those of Affiliated Hospital). 2,475 are faculty members, including 450 professors. 54\% of them have got Ph.D degrees and over 24\% have experience of overseas study. The current total enrollment of full-time students amounts to over 33,000, including 10,000 postgraduates, 1000 international students from 74 countries. Jingjiang College of Jiangsu Univesity has an enrollement of about 10000 full-time students.

JSU has been promoting high-level research. In the recent 5 years, the total scientific research fund amounts to 2.789 billion RMB, sponsored by the governments and enterprises. The number of authorized is ranked 6 among China’s universities. Five disciplines have been ranked as top 1\% in ESI, such as Engineering, Clinical Medicine, Materials Science, Chemistry and Agricultural Science. Drawing on the big varieties of programs and multi-disciplinary strengths, we operate an array of research institutes and centers serving as both the academic think tanks and technological innovation source at the national and regional levels. 

JSU gives priorities to the internationalization of the schooling, encouraging faculty members and students to go abroad for further studies, inviting more global talents to join us for mutual benefits, promoting international research collaboration as well as recruiting more overseas students.

JSU has signed institutional cooperation agreements with 128 universities in 33 countries and regions by May 2017. The Confucius Institute co-built by JSU and Graz University (Austria) has been operating smoothly since October 2010, followed by the opening of the collaborative Chinese-German Language and Culture Center.

\englishkeywords{\large UJS, master thesis, XeTeX/LaTeX template}
\end{englishabstract}

