%# -*- coding: utf-8-unix -*-
%%==================================================
%% conclusion.tex for UJSThesis
%% Encoding: UTF-8
%%==================================================

\begin{summary}

这里是全文总结内容。

为提高我校新一届处级领导干部的综合素质,更好地履行岗位职责,为学校“双一流”和高水平研究型大学建设作出更大的贡献。7月4日至7月5日,学校举办了处级领导干部培训班,近三百名处级领导干部参加了本次培训。全体校领导出席了培训班。

开班典礼上,校党委书记袁寿其作了开班动员。他指出,在“双一流”和江苏高水平研究型大学建设的背景下,干部的履职能力、工作水平与成长进步对学校事业发展至关重要。希望处级领导干部保持良好的精神状态,认真听取报告,学有所思、学有所悟,提高分析问题解决问题的能力,为“双一流”背景下我校高水平研究型大学建设贡献力量。开班典礼由校长颜晓红主持。

此次培训为期二天,共设专家报告、专题发言、分组讨论及培训总结四个环节。武书连、程莹、张巘等专家分别作了专题报告。副校长程晓农、陈龙、施卫东、梅强分别就“双一流”背景下我校科技工作、师资队伍建设、学科建设与研究生教育、本科人才培养等发展思路作了专题发言。全体处级领导干部分为11个组就“双一流”背景下我校高水平研究型大学发展战略进行了热烈讨论。      

在培训总结会上,校长颜晓红就“双一流”背景下学校发展战略,特别是发展动力问题进行了阐述。他指出,我们要以特色鲜明为引领,以创新驱动为发展动力,提高创新能力水平,以改革、国际化为助推器,助推我校“双一流”和高水平研究型大学的建设。

校党委书记袁寿其在总结讲话中强调,领导干部要充分认识学校事业发展所面临的新形势新机遇,认清我校“双一流”建设和高水平研究型大学建设的前期工作和现状,以培养一流的人才、集聚一流的师资队伍、形成一流的学科、产出一流的成果等为目标,牢记使命、明确职责,统一思想、充满信心,真抓实干、勇于担当,通过“干、实干、努力干、创造性干、竭尽全力干、领导带头干”,不断破解“双一流”和高水平研究型大学建设的瓶颈问题,努力为学校事业发展作出更大的贡献。

全体正(副)调研员、科级领导干部和校各民主党派负责人参加专题报告会。

\end{summary}
